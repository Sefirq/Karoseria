\documentclass{article}
\usepackage{polski}
\usepackage[utf8]{inputenc}
\usepackage[OT4]{fontenc}
\usepackage{graphicx,color}
\usepackage{url}
\usepackage[pdftex,hyperfootnotes=false,pdfborder={0 0 0}]{hyperref}
\usepackage{float}

\begin{document}
\input{_tytulowa}

\section*{Udział autorów}
\begin{itemize}
\item Sebastian Firlik //TODO
\item Adam Pioterek //TODO
\end{itemize}

\section{Filtry i funkcje}
Do wykonania zadania wykorzystane zostały następujące filtry:
\begin{description}
\item[gaussian] — przeprowadza rozmycie Gaussa;\\
na wejściu otrzymuje zdjęcie w odcieniach szarości (macierz 2D) i odchylenie standardowe, na wyjściu daje macierz 2D – przefiltrowane zdjęcie.

//TODO
\end{description}
oraz funkcje:
\begin{description}
\item[ConvexHull] — otoczka wypukła;\\
na wejściu otrzymuje zbiór punktów, na wyjściu zwraca indeksy punktów należących do otoczki.

//TODO
\end{description}
 
\section{Cel eksperymentu}
//TODO
\section{Dane do eksperymentu}
//TODO pochodzenie danych i przykładowe obrazy
\section{Efektywność}
//TODO dyskusja skuteczności (trafność klasyfikowania, zdolność dyskryminacji, problemy) i efektywności (on-line/off-line, fps, itp.)

\end{document}
