\documentclass{article}
\usepackage{polski}
\usepackage[utf8]{inputenc}
\usepackage[OT4]{fontenc}
\usepackage{graphicx,color}
\usepackage{url}
\usepackage[pdftex,hyperfootnotes=false,pdfborder={0 0 0}]{hyperref}
\usepackage{float}

\begin{document}
\thispagestyle{empty} %bez numeru strony

\begin{center}
{\large{Sprawozdanie z laboratorium:\\
Komunikacja człowiek–komputer}}

\vspace{3ex}

Przetwarzanie obrazu — aplikacja

\vspace{3ex}
{\footnotesize\today}

\end{center}


\vspace{10ex}

Prowadzący: dr hab.~inż. Maciej Komosiński

\vspace{5ex}

Autorzy:
\begin{tabular}{lllr}
\textbf{Sebastian Firlik} & inf122485 & sebastian.firlik@student.put.poznan.pl \\
\textbf{Adam Pioterek} & inf122446 & adam.pioterek@student.put.poznan.pl \\
\end{tabular}

\vspace{5ex}

Zajęcia czwartkowe, 15:10.

\vspace{35ex}

\noindent Oświadczam/y, że niniejsze sprawozdanie zostało przygotowane wyłącznie przez powyższych autora/ów,
a wszystkie elementy pochodzące z innych źródeł zostały odpowiednio zaznaczone i~są cytowane w bibliografii.  

\newpage



\section*{Udział autorów}
\begin{itemize}
\item Sebastian Firlik //TODO
\item Adam Pioterek //TODO
\end{itemize}

\section{Filtry i funkcje}
Do wykonania zadania wykorzystane zostały następujące filtry:
\begin{description}
\item[gaussian] — przeprowadza rozmycie Gaussa;\\
na wejściu otrzymuje zdjęcie w odcieniach szarości (macierz 2D) i odchylenie standardowe, na wyjściu daje macierz 2D – przefiltrowane zdjęcie.

//TODO
\end{description}
oraz funkcje:
\begin{description}
\item[ConvexHull] — otoczka wypukła;\\
na wejściu otrzymuje zbiór punktów, na wyjściu zwraca indeksy punktów należących do otoczki.

//TODO
\end{description}
 
\section{Cel eksperymentu}
//TODO
\section{Dane do eksperymentu}
//TODO pochodzenie danych i przykładowe obrazy
\section{Efektywność}
//TODO dyskusja skuteczności (trafność klasyfikowania, zdolność dyskryminacji, problemy) i efektywności (on-line/off-line, fps, itp.)

\end{document}
